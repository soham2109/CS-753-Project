\documentclass[14pt,a4paper]{article}
\usepackage[utf8]{inputenc}
\usepackage{amsmath}
\usepackage{amsfonts}
\usepackage{amssymb}
\usepackage{graphicx}
\usepackage[left=2cm,right=2cm,top=2cm,bottom=2cm]{geometry}
\usepackage[
    backend=bibtex,
    style=authoryear
]{biblatex}
\bibliography{../biblio.bib}
\usepackage{hyperref}
\hypersetup{
    colorlinks=true,
    linkcolor=cyan!60!white,
    filecolor=magenta,
    urlcolor=cyan,
    pdftitle={Sharelatex Example},
    bookmarks=true,
    pdfpagemode=FullScreen,
}

%------------%
% Author Info
%------------%
\author{
  \hspace*{1.5em} Soham Naha\\
  \hspace*{1.5em} \texttt{193079003}
  \and
  Mohit Agarwala\\
  \texttt{19307R004}
  \and
  Abhinav Goud Bingi\\
  \texttt{180050002}\\[1.5cm]
}
%------------%
% Title Info
%------------%
\title{\textsc{\LARGE Indian Institute of Technology}\\[1.5cm] % Name of your university/college
\begin{center}
\includegraphics[height=6cm,width=6cm]{../images/iitb_logo.png}
\end{center}
\textsc{\Large CS 753 Project Report}\\[0.5cm]
\HRule \\[0.4cm]
{ \huge \bfseries Speech to Sign-Language(with emotions) for the Hearing-Impaired}\\[0.4cm] % Title of your document
\HRule \\[1.5cm]
\textbf{Reported By}}
%------------%
% Date Info
%------------%
\date{{\large \today}\\[2cm]}
\newcommand{\HRule}{\rule{\linewidth}{0.5mm}}


\begin{document}
\maketitle

\newpage
\section{Problem Statement and Motivation}
A speech and/or hearing impaired person communicates with others using Sign-Language (SL), that is different from the speech modes of communication. There are a lot of literatures that deal with the problem of conversion from Sign-Language (predominantly American Sign Language or ASL) letters to speech, but the reverse domain is not much explored.

\subsection{Problem Statement}
We choose to explore the domain from speech to ASL. So, our primary task was, given a speech utterence from a speaker convey a message to a person who is hearing-impaired and/or voiceless, then speech has to be converted in a form that the other person can understand, i.e. in a sign language.

\subsection{Motivation}
Most of the literature that is present regarding the field of SL and Speech are related to the field of SL-to-Speech conversion with limited vocabulary datasets.

But research related to the other direction, i.e. from Speech-to-SL is not much explored. In this project, we thus aimed to explore the domain from Speech to Sign-Language with the help of Automatic Speech Recognition. 
This could be used as a conversation model for the speech and/or hearing impaired to interact freely with people who don't have knowledge of SL.

It is also noticeable that conversion of speech-to-SL does not necessarily contain the emotion conveyed through the utterence. So, we also tried to add this emotion part along with the other decodings.


\section{Task at Hand}
In order to convert from Speech-to-SL along with emotions, we tried to follow the approach as stated below:
\begin{itemize}
	\item Convert Speech to English Text
	\item Guess the Emotion conveyed within the utterence in parallel to text estimation
	\item Use the predicted text to estimate the sign-language patterns.
\end{itemize}
So, we modularized the complete pipeline into different blocks that can be trained and modelled independently and integrated again. The modules that we have are thus:
\begin{itemize}
	\item Speech2text
	\item Emotion Detection from Audio
	\item Text2ASL
\end{itemize}


\end{document}